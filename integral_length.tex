\documentclass[12pt]{article}
\usepackage{amssymb,latexsym,amsmath}

\newif\ifpdf
\ifx\pdfoutput\undefined
\pdffalse % we are not running PDFLaTeX                                         
\else
\pdfoutput=1 % we are running PDFLaTeX                                          
\pdftrue
\fi
\ifpdf
\usepackage[pdftex]{graphicx}
\else
\usepackage{graphicx}
\fi
\ifpdf
\DeclareGraphicsExtensions{.pdf, .jpg, .tif}
\else
\DeclareGraphicsExtensions{.eps, .jpg}
\fi



\textwidth = 6.5 in
\textheight = 9 in
\oddsidemargin = 0.0 in
\evensidemargin = 0.0 in
\topmargin = 0.0 in
\headheight = 0.0 in
\headsep = 0.0 in
\parskip = 0.2 in
\parindent = 0.0 in


%%%%%%%%%%%%%%%%%%%%%%%%%%%%%%%%%%%%%%%%%%%%%%%%%%%%%%%%%%%%%%%%%%%%%%%%


\newcommand{\polyf}{\mathcal{P}_d}
\newcommand{\poly}{\mathcal{P}}
\newcommand{\polyfe}{\mathcal{P}_{d+e}}

\DeclareMathOperator{\Span}{span}


\title{Integral Length Scale in the LANL DNS code}
\author{Mark Taylor}

\begin{document}
%\maketitle

The normalization used for the 1D spectrum $E_{ij}$ in our DNS code
is:
\[
\frac12 \int u^2 = \frac12 \sum_k u_k^2  = \sum_k E_{11}(k)
\]
so that
\[
E_{11}(k) =  \frac12 u_k^2
\]
With this normalization, Plancheral's equation becomes:
\[
\int f g = \sum_k f_k g_k
\]

The correlation function is:
\[
R(r) = \frac{\int u(x) u(x+r) \,dx}{\int 1 \,dx   } = 
\int u(x) u(x+r) \,dx
\]
(since the area of our domain is 1).  Applying Plancheral,
gives
\[
R(r) =  \sum_k u_k^2  e^{2 \pi i k  r}
\]
and thus
\[
\int_0^1 R(r) \,dr = \sum_k u_k^2   \int_0^1 e^{i k 2 \pi r} \,dr
                   = u_0^2 = 2 E_{11}(0)
\]
So using the definition
\[
L_{11} = \frac{\int_0^1 R(r) \,dr}{< u^2 > }
\]
we have
\[
L_{11} =  \frac{ 2 E_{11}(0) }{ 2 \sum_k E_{11}(k) }
 = \frac{ E_{11}(0) }{ \sum_k E_{11}(k) }
\]


\end{document}