\documentclass[12pt]{article}
\usepackage{amssymb,latexsym,amsmath,bm}

\newif\ifpdf
\ifx\pdfoutput\undefined
\pdffalse % we are not running PDFLaTeX                                         
\else
\pdfoutput=1 % we are running PDFLaTeX                                          
\pdftrue
\fi
\ifpdf
\usepackage[pdftex]{graphicx}
\else
\usepackage{graphicx}
\fi
\ifpdf
\DeclareGraphicsExtensions{.pdf, .jpg, .tif}
\else
\DeclareGraphicsExtensions{.eps, .jpg}
\fi



\textwidth = 6.5 in
\textheight = 9 in
\oddsidemargin = 0.0 in
\evensidemargin = 0.0 in
\topmargin = 0.0 in
\headheight = 0.0 in
\headsep = 0.0 in
\parskip = 0.2 in
\parindent = 0.0 in


%%%%%%%%%%%%%%%%%%%%%%%%%%%%%%%%%%%%%%%%%%%%%%%%%%%%%%%%%%%%%%%%%%%%%%%%


\newcommand{\khat}{\hat{\mathbf k}}
\newcommand{\uv}{\mathbf u}
\newcommand{\up}{\mathbf u'}
\newcommand{\w}{\mathbf w}
\newcommand{\grad}{\nabla}
\newcommand{\curlp}{\gradp \times}
\newcommand{\curl}{\grad \times}
\newcommand{\gradp}{\nabla'}
\newcommand{\bk}{\bm{\kappa}}
\newcommand{\kp}{\kappa}

\DeclareMathOperator{\Span}{span}

\title{Transfer function in the Boussinesq equations}
\author{Susan Kurien}
\begin{document}
\section{Preliminaries}
We would like to compute the energy transfer as a function of
wavenumber in the Boussinesq equations which are given by:
\begin{eqnarray}
\frac{\partial}{\partial t}{u}_j + u_i \partial_i u_j + f\epsilon_{j3k}u_k +
\frac{1}{\rho_0}\partial_j p + \delta_{j3}\frac{\tilde\rho}{\rho_0}g\delta_{j3} &=&
\nu \partial_i^2 u_j\label{momentum}\\ \frac{\partial}{\partial t}\tilde\rho+u_i \partial_i \tilde\rho  -
bu_3 &=& \gamma \partial_i^2 \tilde \rho \label{density}\\ \partial_i u_i &=& 0,
\label{bous}
\end{eqnarray}
in a cartesian coordinate system with unit vectors ($\hat{\bm{x}_1},
\hat{\bm{x}_2}, \hat{\bm{x}_3}$); where $\bm{u}$ is the velocity, $u_3$ is its (vertical) component along $\hat{\bm{x}_3}$, $p$ is the pressure,
$f=2\Omega$ is the Coriolis parameter, $\Omega$ is the constant
background rotation rate, gravitational accelaration $g$ acts in the
vertical direction $\hat{\bm{x}_3}$, $\nu = \mu/\rho_0$ is the
viscosity and $\gamma$ is the mass diffusivity coefficient.  The total
density is $\rho(\bm{x}) = \rho_0 - bx_3 + \tilde\rho(\bm{x})$, where
$\rho_0$ is the constant background, $b$ is also constant and larger
than zero for stable stratification, and $\tilde \rho$ is the density
fluctuation such that $\tilde \rho \ll bw \ll \rho_0$.

Let us derive the energy spectral distribution dynamics first directly
from the fourier representation of the momentum-density equations and
then from the physical space correlation function equations.

\section{Fourier transform of the momentum-density equations}

Let us denote the fourier transform operator by
$\cal{F}_{\bk}$ (using Pope's notation). The fourier transform of a function is denoted by the tilde above the corresponding variable. Term by term the fourier transform of Eq. \ref{momentum} is:
\begin{eqnarray} 
{\cal F}_{\bk}\{\partial_t u_j\} &=& \frac{d}{dt} \tilde u_j(\bm{\kappa},t) =
\frac{d}{dt}\sum_{\bm{k}} e^{i\bm{\kappa\cdot x}} u_j(\bm{x},t)\\
{\cal F}_{\bk}\{u_i \partial_i u_j\} &=& \tilde G_j(\bm{\kappa})\\
{\cal F}_{\bk}\{f\epsilon_{j3k}~u_k \} &=& f\epsilon_{j3k}\tilde u_k(\bm{\kappa})\\
{\cal F}_{\bk}\{\frac{1}{\rho_0}\partial_j p\} &=& \frac{-i}{\rho_0} \kappa_j \tilde p \\
{\cal F}_{\bk}\{\frac{\tilde\rho}{\rho_0}g \delta_{j3}\}&=& \frac{g \delta_{j3}}{\rho_0}\tilde{\tilde \rho}\\
{\cal F}_{\bk}\{\nu \partial_i^2 u_j\}& =& -\nu\kappa^2 \tilde u_j
\end{eqnarray}
Hence the fourier representation of the momentum equation is:
\begin{eqnarray}
\frac{d}{dt} \tilde u_j(\bm{\kappa},t) + \tilde G_j(\bm{\kappa}) + f\epsilon_{j3k}\tilde u_k(\bm{\kappa}) - \frac{i}{\rho_0} \kappa_j \tilde p +  \delta_{j3}\frac{g\delta_{j3}}{\rho_0}\tilde{\tilde \rho} =  -\nu\kappa^2 \tilde u_j
\label{momentum-f}
\end{eqnarray}
Similarly, the density equation and the incompressibility may be written as:
\begin{eqnarray}
\frac{d}{dt} \tilde {\tilde \rho} + \tilde J(\bm{\kappa}) - b \tilde u_3 = \gamma \kappa^2 \tilde {\tilde \rho}\label{density-f}\\
\kappa_i \cdot \tilde u_i \label{incomp-f} = 0
\end{eqnarray}
where ${\cal F}_{\bm{\kappa}}\{u_i \partial_i \tilde\rho\} = \tilde J(\bm{\kappa})$, the $\tilde \tilde$ on the $\rho$ denotes fourier transform of the density fluctuations.

Note that $\tilde G_j(\bm{\kappa})$ and $\tilde J(\bm{\kappa})$ are the fourier
transforms of the non-linear terms and may be simplified to diadic
interactions (see eg. Pope chapter 6 for $\tilde G_j$) but we won't do that here since we
can directly compute the fourier transforms of the entire term.

Next, lets write down the projection operator for the pressure.
Left-multiply Eq. (\ref{momentum-f}) by $\kp_j$ and using Eq. (\ref{incomp-f}):
\begin{eqnarray}
\kp_j\tilde G_j(\bk) &-& f \tilde \omega_3 - \frac{i}{\rho_0} \kp^2 \tilde p +  
\frac{g\delta_{j3}}{\rho_0}\kp_3\tilde{\tilde \rho} = 0\nonumber\\
\mbox{that~is,}~~\tilde p &=& -i\frac{\rho_0}{\kp^2}
\Big(\kp_j\tilde G_j(\bk) - f \tilde \omega_3 
+  \frac{g\delta_{j3}}{\rho_0}\kp_3\tilde{\tilde \rho}\Big) 
\label{k-momentum-f}
\end{eqnarray}
where $\tilde\omega_3=\epsilon_{3jk}\kp_j \tilde u_k(\bk)$ is the
vertical component of the fourier transform of the vorticity. Note the contribution to the pressure from rotation and stratification. Substitute (\ref{k-momentum-f}) back into (\ref{momentum-f}) (keep track of dummy indices!!)..
\begin{eqnarray}
\frac{d}{dt} \tilde u_j(\bm{\kappa},t)  + (\delta_{ij}-\frac{\kappa_j \kp_i}{\kp^2}) \tilde G_i + (\delta_{j3} - \frac{\kp_j \kp_3}{\kp^2})\frac{g\delta_{j3}}{\rho_0}\tilde{\tilde \rho} =  -\nu\kappa^2 \tilde u_j
\label{momentum-f-elimp}
\end{eqnarray}
THUS: when pressure is eliminated using incompressibility, the rotation terms disappear from the momentum equation.

\subsection{Energy spectrum evolution}
The mean energy in wavenumber $\bk$ is given by:
\begin{equation}
E(\bk) = \langle \tilde{{u}_j}(\bk) \tilde{{u}_j}^*(\bk) \rangle = \langle \tilde{{u}_j}(\bk) \tilde{{u}_j}(-\bk) \rangle= \langle |\tilde{\bm{u}}(\bk)|^2 \rangle
\end{equation}
where $\langle \cdot  \rangle$ denotes the average.
So, left-multiply Eq. (\ref{momentum-f-elimp}) by
$\tilde{u_j}^*(\bk)$ and add it to the equation for $\tilde{u_j}^*$
left-multiplied by $\tilde{u_j}$ (the $\bk$ argument is implicit everywhere and the $\delta_{j3}$ for the direction of $g$ will be dropped from now on):
\begin{equation}
\frac{d}{dt} \langle\tilde u_j \tilde{{u}_j}^*\rangle + 
\langle \tilde{u_j}^* \tilde G_j + \tilde{u_j} \tilde G_j^*\rangle+ \langle \tilde{u_3}^*\frac{g}{\rho_0}\tilde{\tilde \rho}  + \tilde u_3 \frac{g}{\rho_0}
\tilde{\tilde \rho}^*\rangle  = -\nu \kp^2 \langle\tilde u_j \tilde u_j^*\rangle
\label{ke}
\end{equation}
The potential energy in wavenumber $\bk$ is obtained by multiplying
(\ref{density-f}) by $\frac{g }{b \rho_0} \tilde{\tilde\rho}^*$ and adding to the equation for 
$\frac{g}{b \rho_0}\tilde{\tilde\rho}^*$ multiplied by $\tilde{\tilde\rho}$:
\begin{eqnarray}
\frac{d}{dt}( \frac{g }{b \rho_0} \tilde{\tilde \rho} \tilde{\tilde \rho}^*)  + 
\frac{g }{\rho_0} \tilde{\tilde \rho}^* \tilde J(\bm{\kappa}) + \frac{g}{\rho_0} \tilde{\tilde \rho} \tilde J^*(\bm{\kappa}) 
- \frac{g}{\rho_0} \tilde u_3 \tilde{\tilde \rho}^* - \frac{g }{\rho_0} \tilde u_3^* \tilde{\tilde \rho} = \gamma \frac{g }{b\rho_0} \kappa^2 (\tilde u_3^* \tilde {\tilde \rho} + \tilde u_3 \tilde {\tilde \rho}^*)
\label{pe}
\end{eqnarray}
 
The total energy evolution is given by the sum of Eqs.~\ref{ke} and \ref{pe}. The transfer terms are:
\begin{itemize}
\item Kinetic-to-kinetic term: $T_K(\bk)=\tilde{u_j}^* \tilde G_j + \tilde{u_j} \tilde G_j^*$

\item Potential-to-potential term: $T_P(\bk)=\frac{g }{\rho_0} (\tilde{\tilde \rho}^* \tilde J +  \tilde{\tilde \rho} \tilde J^*)$

\item Exchange term: $T_E(\bk)=\frac{g}{\rho_0}(\tilde{u_3}^*\tilde{\tilde \rho}  + \tilde u_3 \tilde{\tilde \rho}^*)$
\end{itemize}again, $\tilde G_j(\bk) = {\cal F}_{\bk}\{u_i \partial_i u_j\}$, $\tilde J(\bk) = {\cal F}_{\bm{\kappa}}\{u_i \partial_i \tilde\rho\}$.
This should be enough for our calculations.

\section{The connection to physical space (two-point correlation function) representation}

watch this space.

\end{document}
